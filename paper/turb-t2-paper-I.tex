\documentclass[useAMS, usenatbib, a4paper]{mnras}
\pdfsuppresswarningpagegroup=1
\usepackage[spanish,es-minimal,english]{babel}
\usepackage[utf8]{inputenc}
\usepackage{graphicx}
\let\Bbbk\relax
\usepackage{amsmath}	% Advanced maths commands
\usepackage{amssymb}	% Extra maths symbols
\usepackage{xcolor}
\usepackage{fixltx2e}
\usepackage{hyperref}
\usepackage{savesym}
\savesymbol{tablenum}
\usepackage{siunitx}
\restoresymbol{SIX}{tablenum}
\usepackage{newtxtext}
\usepackage[varg,varvw,smallerops]{newtxmath}
\usepackage{xfrac} % for the \sfrac macro
\usepackage{booktabs}
\usepackage{longtable}
\usepackage{array}   % for \newcolumntype macro
\newcolumntype{L}{>{$}l<{$}} % math-mode version of lrc column types
\newcolumntype{R}{>{$}r<{$}} 
\newcolumntype{C}{>{$}c<{$}}
\hypersetup{colorlinks=True, linkcolor=blue!50!black, citecolor=black,
  urlcolor=blue!50!black}
\usepackage{etoolbox}
\robustify\bfseries
\robustify\itshape
\usepackage{enumerate}
\bibliographystyle{mnras}
\sisetup{
  % explicit""+" is useful for velocities
  retain-explicit-plus = true,
  % prefer 10^6 over 1 x 10^6
  retain-unity-mantissa = false,
  % Use x +/- e instead of x(e)  
  separate-uncertainty = true,
  % Make sure to pick up bold font when used in section heading for instance
  detect-weight = true,
  table-align-uncertainty = true,
  table-align-comparator = true,
}
\DeclareSIUnit\msun{\text{M\ensuremath{_\odot}}}
\DeclareSIUnit\lsun{\text{L\ensuremath{_\odot}}}
\DeclareSIUnit\zsun{\text{Z\ensuremath{_\odot}}}
\DeclareSIUnit\angstrom{\text{\AA}}

% A better \ion command that works in more circumstances
\newcommand\ION[2]{#1\,\scalebox{0.9}[0.8]{\uppercase{#2}}}
\newcounter{ionstage}
\renewcommand{\ion}[2]{\setcounter{ionstage}{#2}% 
  \ensuremath{\mathrm{#1\,\scriptstyle\Roman{ionstage}}}}
\newcommand\hii{\ion{H}{2}}

% Stars in Orion, e.g., \th1C, \th2A
\def\th#1#2{\ensuremath{\theta^{#1}\,\text{Ori~#2}}}
% Wavenumbers
\newcommand\wn{\ensuremath{\tilde{\nu}}}
\newcommand\snrj{SNR~J\num{0059.4}\num{-7210}}

% Chemical formulae
\newcommand*\chem[1]{\ensuremath{\mathrm{#1}}}
% Atomic term symbols
\newcommand\Config[1]{\ensuremath{\mathrm{#1}}}
\newcommand\Term[3]{\ensuremath{\mathrm{#1\ ^{#2}#3}}}
\newcommand\Level[4]{\ensuremath{\mathrm{#1\ ^{#2}#3_{#4}}}}

% Emission lines
\newcommand\ha{\ensuremath{\text{H}\alpha}}
\newcommand\hb{\ensuremath{\text{H}\beta}}
\newcommand\lya{\ensuremath{\text{Ly}\alpha}}
\newcommand\lyb{\ensuremath{\text{Ly}\beta}}

\newcommand\wav[1]{\ensuremath{\lambda #1}}
\newcommand\wavv[1]{\ensuremath{\lambda\lambda #1}}


% Alternative title: Scaling laws for turbulent H II regions
\title[Temperature fluctuations in turbulent H II regions I.]
{
  Temperature fluctuations in turbulent H II regions\\
  I. Large-scale radiative shocks
}

\author[Henney et al.]{
  William J. Henney\textsuperscript{1}\thanks{w.henney@irya.unam.mx},
  J. García-Vázquez\textsuperscript{2},
  S. J. Arthur\textsuperscript{1},
  Sac-Nicté X. Medina\textsuperscript{3},
  and others?
  \\
  \textsuperscript{1}\foreignlanguage{spanish}{%
    Instituto de Radioastronomía y
    Astrofísica, Universidad Nacional Autónoma de México, Apartado
    Postal 3-72, 58090 Morelia, Michoacán, Mexico}\\
  \textsuperscript{2}\foreignlanguage{spanish}{%
    Escuela Superior de Física y Matemáticas,
    Instituto Politécnico Nacional,
    U.P. Adolfo López Mateos, Zacatenco,
    Ciudad de México, México C.P. 07738}\\
  \textsuperscript{3}German Aerospace Center,
  Scientific Information, 51147, Cologne, Germany;
  Max-Planck-Institut für Radioastronomie, Auf dem Hügel 69, 53121, Bonn, Germany\\
}


% These dates will be filled out by the publisher
\date{Accepted XXX. Received YYY; in original form ZZZ}

% Enter the current year, for the copyright statements etc.
\pubyear{2024}


\begin{document}
\label{firstpage}
\pagerange{\pageref{firstpage}--\pageref{lastpage}}
\maketitle



\begin{abstract}
Turbulence can cause temperature fluctuations via two mechanisms in \hii{} regions, via a direct mechanism, and by an indirect mechanism. In the direct mechanism, dissipation of the turbulent kinetic energy acts as a fluctuating heating source for the gas. In the indirect mechanism, the shocks cause density fluctuations, which modulate the ionizing flux that arrives at the outer parts of the \hii{} region, causing the boundary to move in and out, transitioning between a recombination front and an ionization front. If the modulation timescale corresponds to the recombination timescale, then a portion of the gas is out of thermal equilibrium.
\end{abstract}

\begin{keywords}
HII regions -- ISM: kinematics and dynamics -- turbulence 
\end{keywords}
%\facilities{VLT:Yepun (MUSE); OANSPM:2.1m (Mezcal); Keck (HIRES)}
%\object{M42}

\section{Introduction}
\label{sec:introduction}
\begin{figure*}
  \centering
  \includegraphics[width=\linewidth]{figs/kb512-ke-diss-multipanel}
  \caption{Structure of a single-star \hii{} region
    from a three-dimensional radiation-hydrodynamics simulation \citep{medina2014}.
    (a) Plane \(xy\) cut through the fields of gas density (red), squared ionized density (green), and gas temperature (blue).
    (b) Vector velocity field for the same cut: red, green, blue show \(u_x\), \(u_y\), \(u_z\),
    with dark colors indicating negative values and light colors positive values. Mid gray indicates zero.
    (c) Simulated surface brightness image of the simulation cube in 3 different optical emission lines: [\ion{N}{2}] \wav{6584} (red), \ha{} \wav{6563} (green), and [\ion{O}{3}] \wav{5007} (blue).
    (d) Same cut plane as (a), but showing the kinetic energy dissipation rate \(\epsilon\) for ionized gas (red) and neutral gas (green), together with the thermal gas pressure (blue).
    (e) Zoom of central portion of panel~(e), with labeling of different features (see test for details).
    (f) Projection along the line of sight of the ionized emission measure \(\int n_i^2 \, dz\) (green) and the integrated kinetic energy dissipation rate \(\int \epsilon_i \, dz\) (purple).
}
  \label{fig:kb512-mosaic}
\end{figure*}


There are reasons to be suspicious of the high temperatures in ionization transition zones
seen in Figure~\ref{fig:kb512-mosaic}a.
The non-local effects of the diffuse ionizing radiation field
are not considered in the simulation
and the atomic physics of the heating and cooling processes are treated in a very approximate way \citep{Henney:2009b}. 

\section{Summary of García-Vázquez et al. (2023)}
\label{sec:summary-garcia}



\section*{Data availability statement}
\label{sec:data-avail-stat}
All data and accompanying analysis programs used in this paper are available
from the github repository \url{https://github.com/will-henney/turb-t2-paper}.

\bibliography{turb-t2-refs}
\appendix
\section{Cloudy models of low-velocity shocks in \hii{} regions}
\label{sec:cloudy-models-low}



% Don't change these lines
\bsp	% typesetting comment
\label{lastpage}

\end{document}

%%% Local Variables:
%%% mode: LaTeX
%%% TeX-master: t
%%% End:
